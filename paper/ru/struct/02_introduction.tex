В последние годы наблюдается стремительный рост интереса к применению методов искусственного интеллекта,
в частности машинного (ML) и глубокого обучения (DL), в сфере анализа финансовых данных и принятия
инвестиционных решений. Прогресс в области обработки естественного языка (NLP) и развитие больших языковых
моделей (LLM) позволили значительно расширить спектр решаемых задач, включая анализ новостных публикаций
и прогнозирование цен активов \parencite{Jiang2023, Halder2022, Kim2023}. Тем не менее, несмотря
на достижения в области архитектур и вычислительных возможностей, многие аспекты применения
LLM в финансовом домене до сих пор остаются исследовательски и инженерно нерешёнными.

Современные финансовые рынки характеризуются высокой волатильностью и мгновенной реакцией
на внешние события. Новостные сообщения, изменения регуляторного ландшафта, публикации
финансовых отчётов и рекомендации аналитиков формируют непрерывный информационный поток,
оказывающий как краткосрочное, так и накопительное воздействие на цены активов.
Традиционные аналитические системы, главным образом опирающиеся на количественные
показатели --- такие как цены открытия и закрытия, объёмы торгов и классические технические
индикаторы --- не в состоянии объяснить все ценовые колебания. Вследствие этого их
прогностическая точность снижается, а инвестиционные стратегии становятся уязвимыми
к внезапным рыночным шокам.

Современные финансовые рынки характеризуются высокой волатильностью, непрерывным потоком информации
и мгновенной реакцией на значимые события. Выпуски новостей, изменения регуляторной политики, корпоративные
отчёты и экспертные комментарии могут оказывать как краткосрочное, так и накопительное воздействие на показатели
рынка. Однако традиционные аналитические системы зачастую не способны эффективно интегрировать и интерпретировать
такую разнородную и быстро меняющуюся информацию, что снижает их прогностическую силу и делает инвестиционные
стратегии менее устойчивыми к рыночным шокам.

Решение этих задач требует не только усовершенствования архитектурных решений, но и разработки интерпретируемого,
масштабируемого и воспроизводимого инструментария, способного учитывать контекстуальные особенности финансовых
текстов, извлекать аспекты, релевантные к целевым активам, и обобщать лексико-семантические закономерности.
В настоящей работе предлагается подход, опирающийся на методологическую синергию тематического моделирования,
аспекто-ориентированного анализа и векторных представлений текстов.

Эволюция архитектур NLP-систем привела к доминированию трансформерных моделей, начиная с появления трансформерной
архитектуры в 2017 году \parencite{vaswani2017attention}. Языковые модели на базе BERT \parencite{devlin2019BERT}
и его адаптаций (например, FinBERT \parencite{Liu2020FinBERT,Yang2020FinBERT,Huang2023FinBERT,Araci2019FinBERT})
стали стандартом де-факто в ряде прикладных задач. Однако большинство моделей были обучены на корпусах,
не отражающих специфику финансового языка, что ограничивает их применимость \parencite{Jiang2023, devlin2019BERT}.
Финансовые тексты обладают высокой плотностью терминов, наличием отраслевого жаргона, аббревиатур
и формализованных шаблонов, требующих специализированного подхода к семантическому анализу.

До сих пор открыты вопросы интеграции LLM с классическими количественными моделями, отсутствия
достаточного количества решений с откртым исходным кодом для финансовой области и ограничений
существующих моделей для обработки длинных текстовых последовательностей (см. \hyperref[sec:models]{Раздел 1.3.1}).
В декабре 2024 года была представлена новая современная модель ModernBERT, способная обрабатывать тексты, длина которых
в 16 раз превышает возможности предыдущих архитектур \parencite{Warner2024ModernBERT, devlin2019BERT}. Эта модель
расширяет возможности анализа с отдельного заголовка или поста до целых новостных статей, пресс-релизов, транскриптов
интервью и аналитических обзоров. Несмотря на это, обработка многостраничных финансовых отчётов (например, 10-Q, 10-K)
остаётся сложной задачей (см. \hyperref[sec:limitations]{Раздел 2.1}).

Ранее в подобных исследованиях акцент делался на обработке коротких текстов, таких как заголовки
новостей или постов в социальных сетях, что неизбежно ограничивало доступный контекст и повышало чувствительность
моделей к экспрессивным и стилистически окрашенным элементам текста. Проблема заключается в том,
что такие форматы текстов могут усиливать эффекты эмоционального заголовка или стилистических
приемов, не отражая реального содержания публикации. В настоящей работе данная проблема
нивелируется за счёт анализа полного текста, что позволяет учитывать как эмоциональные,
так и содержательные характеристики новостей, снижая влияние поверхностных факторов.

Таким образом, если рассматривать проблему со стороны управления, при принятии инвестиционных решений на волатильных рынках,
важно своевременно анализировать совокупное влияние различных событий (новости, законодательные изменения, аналитика и т. д.)
на динамику активов. Специалисты не способны справиться с таким объемом информации за предельно короткие сроки. С другой
стороны, отсутствие инструмента для комплексного анализа приводит к запаздывающим или неточным решениям, что снижает
эффективность инвестиционных стратегий и увеличивает риск упущения возможностей.

Разработка инструмента, который мог бы разрешить данную ситуацию и упростить процесс принятия решений является трудоёмкой
и комплексной задачей, которую можно условно разбить на следующие этапы:

\begin{enumerate}
    \item Разработка эффективной архитектуры для LLM.
    \item Адаптация модели под специфику финансового домена.
    \item Тонкая настройка модели для решения конкретных задач.
    \item Интеграция модели в систему, работающую как с количественными, так и с качественными данными,
    включая этапы обучения, тестирования и внедрения.
\end{enumerate}

Важно отметить, что конечной целью подобного инструмента стоит автономное эффективное финансовое прогнозирование, которое
бы могло быть с легкостью интерпретировано финансовым аналитиком, контролирующим систему. Стоит уточнить, что речь
идет об эффективном финансовом прогнозировании именно с точки зрения теории эффективности рынка (EMT) \parencite{emt1970fama}.

Итак, поскольку базовая архитектура ModernBERT уже разработана, а методы прогнозирования стоимости финансовых активов при помощи нейронных
сетей изучены на достаточном уровне, настоящее исследование фокусируется на разработке инструментария для повышения как эффективности, так и
интерпретируемости финансового прогнозирования. Причем исследование формулирует цель достичь повышения эффективности прогнозирования
не за счет количественного или итеративного улучшения имеющихся методов, а за счет изучения принципиально новых областей и качетсвенного
технологического скачка, заключающегося в новой парадигме финансового прогнозирования. Так, настоящее исследование предлагает
достичь качественного скачка за счет расширения модальности данных, использующихся для прогнозирования, а также
разработки методов их предобработки.

Данная задача, как отмечалось ранее, комплексна и крайне трудна. Поэтому предметом работы является не финансовое прогнозирование,
а инструментарий (комплекс методов и конкретные технологии) для интегации языковых моделей, а также методов тематического
моделирования и анализа тональности в процесс прогнозирования стоимости активов, которые сделают возможным в последующих
исследованиях качественный рывок в целевой задаче.

Так, на данный момент, в финансовом прогнозировании LLM используются в крайне узком спектре, так как сама технология относительно нова.
Тем не менее, уже есть первые попытки использовать языковые модели для анализа тональности новостей и интеграции тональностей в процесс
прогнозирования, причем несмотря на ограниченность ресурсов, методов и технологий данный подход зарекомендовал себя как достаточно
эффективный. В частности, модели, применяющие тональности в финансовом прогнозирвании показывают результаты выше тех, что делают
прогнозы, основываясь исключительно на количественных данных \parencite{Kim2023, Jiang2023, Halder2022}. Однако, применимость текстовой
модальности для финансового прогнозирования на данный момент крайне ограничено и новым направлением, пока еще не нашедшем применение
в финансах является аспекто-ориентированый сентиментальный анализ \parencite{SA2020taxonomy,FSA2020problems} и тематическое моделирование
\parencite{angelov2020top2vec,BERTopic2022}.

Таким образом, объектом исследования являются инвестиционные стратегии, основанные на методах искусственного интеллекта. Цель работы
заключается в создании практико-ориентированного интерпретируемого инструментария, на основе которого станет
возможным разработать систему для динамического мультимодального прогнозирования стоимости актива с использованием иерархического
аспекто-ориентированного сентиментального анализа. Отдельно стоит подчеркнуть, что ключевым условием предложенного решения является
его интерпретируемость в отличие от других решений, использующих глубокие нейронные сети как черный ящик.

На протяжении исследования были выполнены следующие работы:

\begin{itemize}
    \item Анализ и выбор современных и наиболее эффективных архитектур и моделей (см. Разделы \hyperref[sec:ml_algos]{1.2} и
    \hyperref[sec:models]{1.3}) для целевой задачи.
    \item Проектирование концепта прицнипиально новой системной архитектуры для эффективного финансового прогнозирования
    (см. Раздел \hyperref[sec:architecture]{3.1}).
    \item Разработка финансовой аспекто-ориентированной гибридной семантической системы (Financial Aspect-Based hybrid Semantic System, FinABYSS),
    направленной на реализацию необходимых системных комонент для интерпретируемой финансовой аналитики на мультимодальных данных
    (см. Раздел \hyperref[sec:components]{3.2}).
    \item Сбор большого и исчерпывающего корпуса финансовых публикаций размером более 15Гб, пригодного для использования в смежных финансовых исследованиях
    (см. Раздел \hyperref[sec:data_governance]{2.2}).
\end{itemize}

Следует подчеркнуть глобальную значимость данного исследования. Несмотря на то что в качестве
предмета экспериментов выбран рынок США, это обусловлено исключительно изобилием, доступностью
и единообразием открытых данных по данному рынку. Методология и инструментарий, разработанные
в работе, легко масштабируются и адаптируются к любым другим финансовым рынкам. Универсальность
подхода гарантирует его применимость в различных юрисдикциях и на различных уровнях развития
информационной инфраструктуры, что подчёркивает принципиальную глобальность исследования.

Все результаты данного исследования, включая код для сбора данных, обучающий код, результаты анализ данных и моделей, а также сами модели доступны
в официальном GitHub репозитории проекта\footnote{FinABYSS (Financial Aspect-Based hybrid Semantic System) [Electronic resource] //
Томин Д.В. --- 2025 --- URL: \url{https://github.com/denisalpino/FinABYSS} (Дата обращения: 26.05.2025). --- Режим доступа: по запросу.}.
Собранный корпус доступен в репозитории HuggingFace\footnote{YahooFinanceNewsRaw [Electronic resource] //
Томин Д.В. --- 2025 --- URL: \url{https://huggingface.co/datasets/denisalpino/YahooFinanceNewsRaw} (Дата обращения: 20.04.2025). --- Режим доступа: по запросу.}.